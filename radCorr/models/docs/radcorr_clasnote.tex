% preambolo per doppia compilazione HTML/PDF
\ifx\pdfoutput\undefined      % compilazione htlatex
\documentclass{article}
\DeclareGraphicsExtensions{.png, .gif, .jpg}
\newcommand{\href}[2]{\Link[#1]{}{} #2 \EndLink}
\newcommand{\hypertarget}[2]{\Link[]{}{#1} #2 \EndLink}
\newcommand{\hyperlink}[2]{\Link[]{#1}{} #2 \EndLink}
\else                         % compilazione pdflatex
\documentclass{article}
\usepackage{graphicx}
\usepackage{listings}
\usepackage{fancyhdr}
\usepackage{wrapfig}
\usepackage{multirow}
\usepackage{lscape}
\usepackage{slashed}
\usepackage{color}
\usepackage{amssymb,amsmath}
\pdfpagewidth 8.5in
\pdfpageheight 11in
\setlength\textwidth{5.7in}
\setlength\textheight{8.1in}
\setlength\oddsidemargin{0in}
\setlength\evensidemargin{0in}
\setlength\topmargin{-0.6in}
\setlength\footskip{0.6in}
\setlength\headsep{0.6in}
\usepackage[hyperindex]{hyperref}
\newcommand{\percent}{\,^0\!/_0}
 \hypersetup{
    unicode=false,          % non-Latin characters in Acrobat’s bookmarks
    pdftoolbar=true,        % show Acrobat’s toolbar?
    pdfmenubar=true,        % show Acrobat’s menu?
    pdffitwindow=true,      % page fit to window when opened
    pdfauthor={Maurizio},   % author
    pdfsubject={Ungaro},    % subject of the document
    pdfnewwindow=true,      % links in new window
    colorlinks=true,        % false: boxed links; true: colored links
    linkcolor=black,        % color of internal links
    citecolor=blue,         % color of links to bibliography
    filecolor=magenta,      % color of file links
    urlcolor=blue           % color of external links
}
\fi


\begin{document}

\pagestyle{fancy}
\renewcommand{\sectionmark}[1]{\markright{\slshape \thesection\ #1}{}}
\fancyhead[R]{\bf\rightmark}
\fancyhead[L]{Radiative Corrections}
\fancyfoot[R]{ \sl M. Ungaro, K. Joo}
\fancyfoot[L]{ \sl UCONN/JLAB}

\begin{flushright}
CLAS-NOTE 2010-06\\
{\small svn: https://clas12svn.jlab.org/repos/trunk/user/ungaro/analysis/radiative\_correction/models/docs }\\
Version: 1.0
\end{flushright}

\vspace{0.6cm}

\begin{center}
{\Large \bf Meson electro-production Radiative Corrections based on Exclurad} \hfill \\
\vspace{0.6cm}
{  M. Ungaro, K. Joo} \hfill
\end{center}
\vspace{0.6cm}

\abstract{Meson production calculations of radiative corrections using {\it exclurad},
a theoretical framework developed at Jefferson Lab, are described. The missing mass cut
used in the calculations is discussed. }

\tableofcontents

\include{clas_note_radcorr}
\begin{thebibliography}{mybib}
	\bibitem {bib:radcorr}   {A. Afanasev, I. Akushevich, V. Burkert, K. Joo},       {\it PiN Newslett.  {\bf 16}, 343 (2002)}
	\bibitem {bib:radinfra}  {D.Y. Bardin, N.M. Shumeiko},                           {\it Nucl. Phys. B {\bf 127}, 242 (1977)}
	\bibitem {bib:shum}      {N. M. Shumeiko}                                        {\it Sov. J. Nucl. Phys, {\bf 29}, 807 (1979)}
	\bibitem {bib:YFS}       {D. R. Yennie, S. C. Frautschi, H. Suura}               {\it Ann. Phys. {\bf 13} 379 (1961)}
	\bibitem {bib:motsai}    {Mo and Tsai},                                          {\it Rev. Mod. Phys. {\bf 41} (1969)}
\end{thebibliography}


\end{document}








