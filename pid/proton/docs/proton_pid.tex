% preambolo per doppia compilazione HTML/PDF
\ifx\pdfoutput\undefined      % compilazione htlatex
\documentclass{article}
\DeclareGraphicsExtensions{.png, .gif, .jpg}
\newcommand{\href}[2]{\Link[#1]{}{} #2 \EndLink}
\newcommand{\hypertarget}[2]{\Link[]{}{#1} #2 \EndLink}
\newcommand{\hyperlink}[2]{\Link[]{#1}{} #2 \EndLink}
\else                         % compilazione pdflatex
\documentclass{article}
\usepackage{graphicx}
\usepackage{listings}
\usepackage{fancyhdr}
\usepackage{wrapfig}
\usepackage{multirow}
\usepackage{lscape}
\usepackage{amssymb,amsmath}
\pdfpagewidth 8.5in
\pdfpageheight 11in
\setlength\textwidth{5.7in}
\setlength\textheight{8.1in}
\setlength\oddsidemargin{0in}
\setlength\evensidemargin{0in}
\setlength\topmargin{-0.6in}
\setlength\footskip{0.6in}
\setlength\headsep{0.6in}
\usepackage[hyperindex]{hyperref}
\newcommand{\percent}{\,^0\!/_0}
 \hypersetup{
    bookmarks=true,         % show bookmarks bar?
    unicode=false,          % non-Latin characters in Acrobat’s bookmarks
    pdftoolbar=true,        % show Acrobat’s toolbar?
    pdfmenubar=true,        % show Acrobat’s menu?
    pdffitwindow=true,      % page fit to window when opened
    pdfauthor={Maurizio},   % author
    pdfsubject={Ungaro},    % subject of the document
    pdfnewwindow=true,      % links in new window
    colorlinks=true,        % false: boxed links; true: colored links
    linkcolor=black,        % color of internal links
    citecolor=blue,         % color of links to bibliography
    filecolor=magenta,      % color of file links
    urlcolor=blue           % color of external links
}
\fi


\begin{document}
\pagestyle{fancy}
\renewcommand{\sectionmark}[1]{\markright{\slshape \thesection\ #1}{}}
\fancyhead[R]{\bf\rightmark} 
\fancyhead[L]{e1-6 analysis}
\fancyfoot[L]{ \sl JLAB/UCONN/UVA}

\title{\large e1-6 Proton Identification}
 \author{M. Ungaro, V. Burkert, K. Joo, C. Smith, P. Stoler}
\maketitle

\abstract{This document describes the proton identification for the e1-6 data.
The candidate protons are all the events containing at least one positive time-based track, after the electron identification.
The cuts are based on the candidate's reconstructed momentum and its signals on
the Time of Flight (TOF)\cite{bib:ftof}. The cuts are sector-dependent.}

\tableofcontents


\include{e16_p_id}
\include{e16_p_id_bib}

\end{document}








